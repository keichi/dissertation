\chapter{Conclusion}\label{sec:v}

\section{Concluding Remark}
% 問題
The inter-process communication of applications running on cluster systems
show distinctive patterns. However, in contrast to the application-dependent
communication pattern, the interconnect is inherently designed in an
application-agnostic manner because a real-world cluster is usually shared by
many users and each user runs various applications. Therefore, in As a result,
the imbalance of the packet flow in the interconnect can occur under a
particular combination of communication pattern and interconnect. This
imbalance can lead to traffic congestion on a link in the interconnect, which
slows down communication and as a result degrades the total application
performance.

% 目的
This dissertation tackled this imbalance problem by adapting the interconnect
to the communication pattern of applications. Traditionally, such dynamic
adaptation of the interconnect has been deemed infeasible due to the lack of a
networking architecture, technology, or technique that allows flexible and
dynamic reconfiguration. However, the recent emergence of programmable
networking architecture that allows flexible and dynamic reconfiguration.
However, the recent emergence of programmable networking architectures
exemplified by Software-Defined Networking~(SDN) has opened up the possibility
to realize such adaptation. This dissertation aimed to overcome this
shortcoming of conventional application-agnostic interconnects by establishing
a programmable interconnect control that dynamically controls the packet flow
in the interconnect based on the communication pattern of applications.

% 課題
Following three challenges have been tackled to achieve the goal described
above: (1) analyzing the packet flow in the interconnect, (2) accelerating MPI
communication by dynamically controlling the packet flow in the interconnect,
and (3) coordinating the execution of application and interconnect control.

% 提案1
To address the first challenge, Chapter~\ref{sec:ii} proposed PFAnalyzer, a
toolset for analyzing the packet flow in the interconnect. When designing and
implementing an efficient programmable interconnect control, researchers need
to conduct a systematic analysis over many combinations of applications and
interconnects. Since performing such an analysis on a physical cluster is
time-consuming, this research utilizes simulation to facilitate the analysis.
The proposed toolset is a pair of tools: an interconnect simulator
specialized for programmable interconnects, and a profiler to collect
communication pattern from applications. PFSim allows researchers and
designers working on interconnects to investigate possible congestion in the
interconnect for an arbitrary cluster configuration and a set of communication
patterns extracted by PFProf. In the evaluation, the accuracy of the
simulation results obtained from PFSim was assessed. Furthermore, how
PFAnalyzer can be used to analyze the effect of programmable interconnect
control was demonstrated.

% 提案2
To address the second challenge, Chapter~\ref{sec:iii} proposed a framework to
accelerate MPI collectives by dynamically controlling the packet flow in the
interconnect. Message Passing Interface~(MPI) is a standardized inter-process
communication library widely used to develop parallel distributed
applications for clusters. Out of the communication primitives provided by
MPI, this research focused on accelerating collective communication because it
occupies a significant fraction of the execution time of applications. The
network programmability provided by Software-Defined Networking was integrated
into MPI collectives in such a way that MPI collectives are able to
effectively utilize the bandwidth of the interconnect. In particular, this
research aimed to reduce the execution time of MPI\_Allreduce, which is a
frequently used MPI collective communication in many simulation codes. The
speedup of MPI\_Allreduce when using the proposed collective acceleration
framework was evaluated.

% 提案3
To address the third challenge, Chapter~\ref{sec:iv} proposed UnisonFlow, a
software-defined coordination mechanism that performs interconnect control in
synchronization with the execution of applications. In real-world
applications, the communication pattern changes with the execution of
application. Therefore, a mechanism to coordinate packet flow control and
execution of application is essential. UnisonFlow is a kernel-assisted
mechanism that realizes such coordination on a per-packet basis while
maintaining significantly low overhead. Evaluation verified that the
interconnect control can be successfully performed in synchronization with the
execution of the application and the overhead imposed by the coordination
mechanism is small.

\section{Future Work}

% 複数ジョブへの対応
In this dissertation, it was assumed that only a single job is executed on the
cluster. However, a production cluster usually executes multiple jobs
simultaneously. Therefore, the proposed programmable interconnect control
should be enhanced to support multiple concurrent jobs on a cluster. This
enhancement is a challenging task because of the following two reasons. First,
the interconnect control needs to be coordinated with the scheduling of jobs.
In other words, the interconnect control needs to be triggered each time a job
starts and a job exits. Such coordination could be realized by integrating the
job scheduler with the interconnect controller. Second, inter-job interference
of packet flow needs to be considered. The coexistence of multiple jobs on a
single cluster implies that the packet flow generated by different jobs may
share a single link. Under such situation, the packet flow generated by a
communication-heavy job could occupy the interconnect and degrade the
communication performance of other jobs. In fact, researchers have reported a
significant performance variability of jobs caused by the interference of
packet flow between different jobs~\cite{Bhatele2013}. Therefore, the
programmable interconnect control should globally optimize the packet flow in
the interconnect while considering the communication pattern of each job and
the interference between jobs so that all jobs can equally benefit from the
interconnect control.

% スケーラビリティ
Due to the limited scale of the cluster that was available for the
experiments, scalability of the proposed programmable interconnect control has
not been thoroughly investigated. There are mainly two challenges in scaling
out the programmable interconnect control proposed in this research. The first
challenge is the concentration of load on the interconnect controller. As
described in Section~\ref{sec:i-sdn}, a centralized controller oversees the
entire network in the SDN architecture. This design inherently poses a
scalability issue because the amount of work for the controller, such
monitoring the state of the interconnect and exchanging control messages with
the switches dramatically increases with the number of compute nodes in the
cluster. Fortunately, researchers have proposed multi-controller
architecture for SDN\cite{Hu2018}. Under the multi-controller architecture,
a single network is managed by multiple controllers working cooperatively.
The second challenge is the limit of flow entries. The required number of flow
entries increases rapidly with the number of compute nodes composing the
cluster. This problem might be solved by pruning the number of flow entries by
merging redundant flow entries or evicting flow entries from switches that are
rarely matched.
