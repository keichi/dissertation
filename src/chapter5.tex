\chapter{Conclusion}\label{sec:v}

\section{Concluding Remark}

% 問題
The inter-process communication of applications running on cluster cluster
show distinctive patterns. However, in contrast to the application-dependent
communication pattern, the interconnect is inherently designed in an
application-agnostic manner because a real-world cluster is usually shared by
many users. As a result, imbalance of the packet flow in the interconnect can
occur under a particular combination of communication pattern and
interconnect. This imbalance can lead to traffic congestion on a link in the
interconnect, which slows down communication and ultimately degrade the
total application performance.

% 目的
This dissertation tackled this problem by adapting the interconnect to the
communication pattern of applications. Traditionally, such dynamic adaptation
of the interconnect has been deemed infeasible due to the lack of a networking
architecture that allows flexible and dynamic reconfiguration. However, the
recent emergence of programmable networking architectures exemplified by
Software-Defined Networking~(SDN) has opened up the possibility to realize
such adaptation. This dissertation aimed to overcome this shortcoming of
conventional application-agnostic interconnects by establishing a programmable
interconnect control that dynamically controls the packet flow in the
interconnect based on the communication pattern of applications.

% 課題
Following three challenges must be tackled to materialize this concept: (1)
analyzing the packet flow in the interconnect, (2) accelerating MPI
communication by dynamically controlling the packet flow in the interconnect,
and (3) coordinating the execution of application and interconnect control.

% 提案1
In Chapter~\ref{sec:ii},  we proposed PFAnalyzer, a toolset for analyzing the
packet flow in the interconnect to address the first challenge. When designing
and implementing an efficient programmable interconnect control, researchers
need to conduct a systematic analysis over many combinations of applications
and interconnects. Since performing such analysis on a physical cluster is
time-consuming, we utilize simulation to facilitate the analysis. Our proposed
toolset is a pair of two tools: an interconnect simulator specialized for
programmable interconnects, and a profiler to collect communication pattern
from MPI applications. PFSim allows interconnect researchers and designers to
investigate congestion in the interconnect for an arbitrary cluster
configuration and a set of communication patterns extracted by PFProf. We
evaluate the accuracy of the simulation results obtained from PFSim and
demonstrate how PFAnalyzer can be used to analyze the effect of programmable
interconnect control.

% 提案2
In Chapter~\ref{sec:iii}, we proposed a framework that accelerates MPI
collectives by dynamically controlling the packet flow in the interconnect to
address the second challenge. Out of the communication primitives provided by
MPI, we focus on accelerating collective communication because it consumes a
significant fraction of the execution time of applications. We integrate the
network programmability provided by Software-Defined Networking into MPI
collectives in such a way that collectives are able to effectively utilize the
bandwidth of the interconnect. In particular, we aim to reduce the execution
time of MPI\_Allreduce, which is a frequently used MPI collective
communication in many simulation codes. We evaluate the speedup of
MPI\_Allreduce when using our collective acceleration framework.

% 提案3
In Chpater~\ref{sec:iv}, we proposed UnisonFlow, a software-defined
coordination mechanism that performs interconnect control in synchronization
with the execution of applications to address the third challenge. In a
real-world application, the communication pattern changes with the execution
of application. Therefore, a mechanism to coordinate packet flow control and
execution of application is essential. UnisonFlow is a kernel-assisted
mechanism that realizes such coordination in a per-packet basis while
maintaining significantly low overhead. We verify that the interconnect
control can be successfully performed in synchronization with the execution of
the application and evaluate the overhead imposed by the coordination
mechanism.
