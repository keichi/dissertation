\pagestyle{myheadings}
\markright{}
\chapter*{Summary}

% 背景
Recent breakthroughs in science have significantly benefited from high
performance computing~(HPC). The majority of HPC systems today adopt cluster
architecture to achieve their massive scalable computing performance. A
cluster is a system of computers connected through a high-performance network
referred to as an interconnect. In a cluster, processes running on different
compute nodes work collectively by exchanging data and messages with one
another over the interconnect. Therefore, the communication performance of the
interconnect is critically important in determining the total computing
performance of a cluster.

% 問題
The inter-process communication of an application running on a cluster system
shows a distinctive pattern that originates from the mathematical model,
discretization method and parallelization strategy used in the application.
The design of an interconnect could be highly optimized for a representative
application expected on the target system by taking the communication pattern
of the application into account. However, this design approach is infeasible
and unrealistic when designing a real-world cluster, since many users share a
single HPC system and each user runs various applications. Therefore, in
contrast to the application-dependent communication pattern, the interconnect
is inherently designed in an application-independent manner, assuming a
uniform communication pattern between processes. As a result, an imbalance in
the packet flow in the interconnect can occur under a non-uniform
communication pattern. This imbalance can lead to traffic congestion on a link
in the interconnect, which lowers the throughput of communication and degrades
the total application performance as a result.

% 目的
This dissertation tackles this degradation of application performance by
adapting the control of packet flow in the interconnect to the communication
pattern of applications. Until recently, such dynamic adaptation of the
interconnect control has been deemed infeasible due to the lack of a
networking architecture, technology, or technique that allows flexible and
dynamic reconfiguration. However, the recent emergence of programmable
networking architectures exemplified by Software-Defined Networking~(SDN) has
opened up the possibility to realize such adaptation. This dissertation aims
to overcome this shortcoming of conventional application-independent
interconnects described in the last paragraph, by establishing a programmable
interconnect control that dynamically manages the packet flow in the
interconnect based on the communication pattern of applications.

% 課題
This dissertation tackles the following three challenges to achieve the goal
described above: (1) analyzing the packet flow in the interconnect, (2)
dynamic adaptation of the interconnect control, and (3) coordinating the
execution of application and interconnect control. The first challenge is
required to observe and understand the imbalance of packet flow in the
interconnect to perform an effective adaptation of the interconnect control.
The second challenge is required to mitigate the imbalance of packet flow and
improve the performance of inter-process communication. The third challenge is
required since many real-world applications exhibit time-varying communication
patterns and therefore the interconnect control needs to be performed in
accordance with the execution of an application.

% 提案1
To address the first challenge, Chapter~\ref{sec:ii} proposes PFAnalyzer, a
toolset for analyzing the packet flow in the interconnect. When designing and
implementing an efficient programmable interconnect control, researchers need
to conduct a systematic analysis over many combinations of applications and
interconnects. Since performing such an analysis on a physical cluster is
time-consuming, this research utilizes simulation to facilitate the analysis.
The proposed toolset is a pair of tools: an interconnect simulator
specialized for programmable interconnects, and a profiler to collect
communication pattern from applications. PFSim allows researchers and
designers working on interconnects to investigate possible congestion in the
interconnect for an arbitrary cluster configuration and a set of communication
patterns extracted by PFProf. In the evaluation, the accuracy of the
simulation results obtained from PFSim is assessed. Furthermore, how
PFAnalyzer can be used to analyze the effect of programmable interconnect
control is demonstrated.

% 提案2
To address the second challenge, Chapter~\ref{sec:iii} proposes a framework to
accelerate MPI collectives by dynamically controlling the packet flow in the
interconnect. Message Passing Interface~(MPI) is a standardized inter-process
communication library widely used to develop parallel distributed
applications for clusters. Out of the communication primitives provided by
MPI, this research focuses on accelerating collective communication because it
occupies a significant fraction of the execution time of applications. The
network programmability provided by Software-Defined Networking is integrated
into MPI collectives in such a way that MPI collectives are able to
effectively utilize the bandwidth of the interconnect. In particular, this
research aims to reduce the execution time of MPI\_Allreduce, which is a
frequently used MPI collective communication in many simulation codes. The
speedup of MPI\_Allreduce when using the proposed collective acceleration
framework is evaluated.

% 提案3
To address the third challenge, Chapter~\ref{sec:iv} proposes UnisonFlow, a
software-defined coordination mechanism that performs interconnect control in
synchronization with the execution of applications. In real-world
applications, the communication pattern changes with the execution of
application. Therefore, a mechanism to coordinate packet flow control and
execution of application is essential. UnisonFlow is a kernel-assisted
mechanism that realizes such coordination on a per-packet basis while
maintaining significantly low overhead. Evaluation verifies that the
interconnect control can be successfully performed in synchronization with the
execution of the application and the overhead imposed by the coordination
mechanism is small.

Chapter~\ref{sec:v} concludes this dissertation and discusses future works.

\pagestyle{headings}
